%%%%%%%%%%%%%%%%%%%%%%%%%%%%%%%%%%%%%%%%%%%%%%%%%%%%%%%%%%%%%%%%%%
%%%%%%%% ICML 2011 EXAMPLE LATEX SUBMISSION FILE %%%%%%%%%%%%%%%%%
%%%%%%%%%%%%%%%%%%%%%%%%%%%%%%%%%%%%%%%%%%%%%%%%%%%%%%%%%%%%%%%%%%

\def\E{{\bf E}}
\def\S{K}
\def\reals{\mathbb{R}}
\def\tr{\mathrm{tr}}
\def\cS{\mathcal{S}}
\def\cL{\mathcal{L}}
\def\U{\mathcal{U}}
\def\hp{\hat{p}}

% Use the following line _only_ if you're still using LaTeX 2.09.
%\documentstyle[icml2011,epsf,natbib]{article}
% If you rely on Latex2e packages, like most moden people use this:
\documentclass{article}

% For figures
\usepackage{graphicx} % more modern
%\usepackage{epsfig} % less modern
\usepackage{subfigure}
\usepackage{amsmath}
\usepackage{amsthm}
\newtheorem{theorem}{Theorem}
\newtheorem{lemma}{Lemma}

% For citations
\usepackage{natbib}

% For algorithms
\usepackage{algorithm}
\usepackage{amsfonts}
\usepackage{algorithmic}

% As of 2010, we use the hyperref package to produce hyperlinks in the
% resulting PDF.  If this breaks your system, please commend out the
% following usepackage line and replace \usepackage{icml2011} with
% \usepackage[nohyperref]{icml2011} above.
\usepackage{hyperref}

% Packages hyperref and algorithmic misbehave sometimes.  We can fix
% this with the following command.
\newcommand{\theHalgorithm}{\arabic{algorithm}}

% Employ the following version of the ``usepackage'' statement for
% submitting the draft version of the paper for review.  This will set
% the note in the first column to ``Under review.  Do not distribute.''
\usepackage{icml2011}
% Employ this version of the ``usepackage'' statement after the paper has
% been accepted, when creating the final version.  This will set the
% note in the first column to ``Appearing in''
% \usepackage[accepted]{icml2011}


% The \icmltitle you define below is probably too long as a header.
% Therefore, a short form for the running title is supplied here:
\icmltitlerunning{Actively Learning the Crowd Kernel}

\begin{document}

\icmltitle{Supplementary Material: Capturing the Crowd Kernel}
%or Capturing the Crowd Kernel, Actively?


% It is OKAY to include author information, even for blind
% submissions: the style file will automatically remove it for you
% unless you've provided the [accepted] option to the icml2011
% package.
\icmlauthor{Your Name}{email@yourdomain.edu}
\icmladdress{Your Fantastic Institute,
            314159 Pi St., Palo Alto, CA 94306 USA}
\icmlauthor{Your CoAuthor's Name}{email@coauthordomain.edu}
\icmladdress{Their Fantastic Institute,
            27182 Exp St., Toronto, ON M6H 2T1 CANADA}

% You may provide any keywords that you
% find helpful for describing your paper; these are used to populate
% the "keywords" metadata in the PDF but will not be shown in the document
\icmlkeywords{active learning, crowdsourcing, kernels}

\vskip 0.3in


\section{Analysis}

Before we present the proof of Theorem 1, we introduce a natural generalization which will be a convenient abstraction.  We call this relative regression.

\section{Relative regression}

Consider the following online relative regression model.  There is a sequence of examples $(x_1,x_1',y_1),(x_2,x_2',y_2),\ldots,(x_T,x_T',y_T) \in X \times X \times \{0,1\}$, for some set $X \subseteq \reals^d$.  For $w \in \reals^d$, define the relative linear model with $w$ to be,
$$p_t(w) = \frac{w \cdot x_t}{w \cdot x_t + w \cdot x_t'}.$$
The sequence $x_1,x_1',\ldots,x_T,x_T'$ is chosen arbitrary (or even adversarially) in advance; afterwards it is assumed that there is some $w^*\in \reals^d$ such that, $\Pr[y_t=1]=p_t(w^*)$ and that the different $y_t$'s are independent.  It is further assumed that $w^*$ belongs to some convex compact set $W \subset \reals^d$ and that $w \cdot x$ is positive and bounded over $w \in W,x\in X$.  Without loss of generality, by scaling we can require $w\cdot x \in [1,\beta]$ for some $\beta>0$ and every $w \in W,x\in X$.

On the $t$th period, the algorithm outputs $w_t \in W$, then observes $x_t,x_t',y_t$, and finally incurs loss $\ell_t(w_t)$ where $\ell_t(w)=\log 1/p_t(w)$ if $y_t=1$ and $\ell_t(w)=\log 1/(1-p_t(w))$ if $y_t=0$.  The goal of the algorithm is to incur total loss not much larger than $\sum_t \ell_t(w^*)$, the best choice had we known $w^*$ in advance.  

We note that an analogous (and slightly simpler version of) the following lemma can be proven for squared loss.
\begin{lemma}
Let $X,W \subseteq \reals^d$ and suppose that $W$ is compact and convex and $\exists \alpha >0$ such that for all $x \in X$, $w \in W$:
$\|x\|,\|w\|\leq 1$, and $w \cdot x \geq \alpha$.
The for any $\eta>0$ and any $w^0\in W$ and $w^{t+1}=\Pi_W(w^t-\eta \nabla \ell_t(w_t))$,
$$\frac{1}{T}\E\left[\sum_{t=1}^T \ell_t(w_t) -\ell_t(w^*)\right] \leq \frac{\eta}{\alpha^2}+\frac{2}{T\eta \alpha}.$$
\end{lemma}
In particular, for $\eta=\sqrt{2\alpha/T}$, this gives a bound on the right-hand side of $\sqrt{\frac{8}{T\alpha^3}}$.
\begin{proof}
Following the analysis of Zinkevich \cite{Zinkevich03} we consider the potential equal to the squared distance $(w_t-w^*)^2$ and argue that it decreases whenever we have substantial error.
Let $\nabla_t = \nabla \ell_t(w_t)\in \reals^d$, which is,
$$\nabla_t = \frac{x_t+x_t'}{w_t\cdot x_t+w_t\cdot x_t'} - y_t \frac{x_t}{w_t\cdot x_t} - (1-y_t)\frac{x_t'}{w_t\cdot x_t'}.$$
By the triangle inequality $\|\nabla_t\| \leq G$ for $G=\frac{2}{\alpha}$.  Now, as Zinkevich points out, due to convexity of $W$, $(w-\Pi_W(v))^2\leq (w-v)^2$ for any $v \in \reals^d$ and $w \in W$.  Hence,
$$(w^*-w_{t+1})^2 \leq (w^*-w_t+\eta \nabla_t)^2.$$
Thus the {\em decrease} in potential, call it $\Delta_t = (w^*-w_t)^2-(w^*-w_{t+1})^2$, is at least:
\begin{align*}
\Delta_t &\geq (w^*-w_t)^2-(w^*-w_t+\eta \nabla_t)^2 \\
&=2\eta \nabla_t \cdot (w_t-w^*)-\eta^2 \nabla_t^2.
\end{align*}

Next, we consider the quantity,
$\E[ \Delta_t \cdot w^* ]$, where the expectation is taken over the random $y_t$ (fixing $y_1,y_2,\ldots,y_{t-1}$).  By expansion, the expectations is:
$$\frac{w^* \cdot x_t + w^* \cdot x_t'}{w \cdot x_t + w\cdot x_t'} - p_t(w^*) \frac{w^* \cdot x_t}{w_t\cdot x_t} - (1-p_t(w^*))\frac{w^* \cdot x_t'}{w_t\cdot x_t'}.$$
After simple algebraic manipulation, which is difficult to show in two-column format, we have,
\begin{align*}
\E[\Delta_t \cdot w^*]&=-Z_t(p_t(w^*)-p_t(w_t))^2 \text{ where }\\
Z_t &=\frac{(w_t\cdot x_t+w_t\cdot x_t')(w^*\cdot x_t+w^*\cdot x_t')}{(w_t \cdot x_t)(w_t \cdot x_t')}.
\end{align*}

Also note that $\Delta_t \cdot w_t =0$ regardless of $y_t$.  Hence,
$\E[ \Delta_t \cdot w_t] =0$.  Combining these with the fact that we have shown that $\Delta_t \geq 2\eta \nabla_t \cdot(w_t-w^*)-\eta^2 G^2$, gives,

\begin{align*}
\E[\Delta_t] &\geq 2\eta Z_t (p_t(w^*)-p_t(w_t))^2-\eta^2G^2\\
&\geq 2 \eta \frac{w^*\cdot x_t + w^* \cdot x_t'}{w_t\cdot x_t + w_t \cdot x_t'}\frac{(p_t(w_t)-p_t(w^*))^2}{p_t(w_t)(1-p_t(w_t))}-\eta^2G\\
&\geq 2\eta\alpha \frac{(p_t(w_t)-p_t(w^*))^2}{p_t(w_t)(1-p_t(w_t))}-\eta^2G.
\end{align*}
In the last line we have used the fact that $w\cdot x \in [\alpha,1]$ for $w\in W,x\in X$.
Now, by Lemma \ref{lem:appx1} which follows this proof,
$$\ell_t(w_t)-\ell_t(w^*) \leq \frac{(p_t(w_t)-p_t(w^*))^2}{p_t(w_t)(1-p_t(w_t))}.$$
Combining the previous two displayed equations gives,
$$\E[\Delta_t]  \geq 2 \eta\alpha (\ell_t(w_t)-\ell_t(w^*))-\eta^2G.$$
Finally, since the potential $(w_t -w^*)^2>0$, we have $\sum \Delta_t \leq (w_0-w^*)^2 \leq 4.$  Hence,
$$\sum_t \ell_t(w_t)-\ell_t(w^*) \leq \frac{T \eta^2 G+4}{2\eta \alpha}.$$
Substituting $G=2/\alpha$ gives the lemma.
\end{proof}



\begin{lemma}\label{lem:appx1}
Let $p+q=1$ and $p^*+q^*=1$ for $p,p^* \in [0,1]$.  Then,
$$p^* \log \frac{p^*}{p} + q^* \log \frac{q^*}{q} \leq \frac{(p-p^*)^2}{pq}.$$
\end{lemma}
\begin{proof}
By concavity of $\log$, Jensen's inequality implies,
$$p^* \log \frac{p^*}{p} + q^* \log \frac{q^*}{q}  \leq \log \frac{(p^*)^2}{p} + \frac{(q^*)^2}{q}.$$
Simple algebraic manipulation shows that,
$$\frac{(p^*)^2}{p} + \frac{(q^*)^2}{q} = 1 + \frac{(p-p^*)^2}{pq}.$$
Finally, the fact that $\log 1+x \leq x$ completes the lemma.
\end{proof}


\subsection{Proof of Theorem 1}

To prove theorem 1, map matrix $S\in \reals^{n \times n}$ to a vector $w(S)\in\reals^{1+n^2}$ consisting of the constant $\mu+2$ in the first coordinate followed by the $n^2$ entries of $S$.  Taken over the set of symmetric $S \succeq 0$ such that $S_{ii}=1$, the vectors $w(S)$ for a compact convex set of radius $\sqrt{n^2+(2+\mu)^2}$.  Also,
$$p^{a_t}{b_tc_t}=\frac{\mu + 2-\S_{ac}-\S_{ca}}{2\mu + 4-\S_{ab}-\S_{ba}-\S_{ac}-\S_{ca}},$$
is our relative regression model for $w=w(S)$, $x\in\reals^{1+n^2}$ being the vector with a $1$ is the first position and -1's in the positions corresponding to the $ac$ and $ca$ entries of $S$ (and zero elsewhere), and $x'$ having a 1 in the first position and -1's in the positions corresponding to $ab$ and $ba$ (and zero elsewhere).  The inner product of $w(S)$ and $x$ is $\mu+\delta_{ab}$ and hence is bounded.  
To apply Lemma \ref{lem:rel}, one must scale $w(S)$ and $x,x'$ down.  However, it is clear that for any $\eps$ and any $\eta$ sufficiently small, there is a $T_0$ such that for $T\geq T_0$, the expected average regret is at most $\eps$.


We now give some justification for why gradient descent should not get trapped in local minima.  As is sometimes the case in learning, it is easier to analyze an online version of the algorithm, i.e., a stochastic gradient descent.  Here, we suppose that the sequence of triplets is presented in order: the learner predicts $S^{t+1}$ based on $(a_1,b_1,c_1,y_1),\ldots,(a_t,b_t,c_t,y_t)$.  The loss on iteration $t$ is $\ell_t(S^t)=\log 1/p$ where $p$ is the probability that the relative model with $S^t$ assigned to the correct outcome.  One can show that
\begin{theorem}

\end{theorem}


\bibliography{sim}
\bibliographystyle{icml2011}

\end{document}


% This document was modified from the file originally made available by
% Pat Langley and Andrea Danyluk for ICML-2K. This version was
% created by Lise Getoor and Tobias Scheffer, it was slightly modified
% from the 2010 version by Thorsten Joachims & Johannes Fuernkranz,
% slightly modified from the 2009 version by Kiri Wagstaff and
% Sam Roweis's 2008 version, which is slightly modified from
% Prasad Tadepalli's 2007 version which is a lightly
% changed version of the previous year's version by Andrew Moore,
% which was in turn edited from those of Kristian Kersting and
% Codrina Lauth. Alex Smola contributed to the algorithmic style files.


